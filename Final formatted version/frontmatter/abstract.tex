%The word �Abstract� should be centered 2? below the top of the page. 
%Skip one line, then center your name followed by the title of the 
%thesis/dissertation. Use as many lines as necessary. Centered below the 
%title include the phrase, in parentheses, �(Under the direction of  
%_________)� and include the name(s) of the dissertation advisor(s).
%Skip one line and begin the content of the abstract. It should be 
%double-spaced and conform to margin guidelines. An abstract should not 
%exceed 150 words for a thesis and 350 words for a dissertation. The 
%latter is a requirement of both the Graduate School and UMI's 
%Dissertation Abstracts International.
%Because your dissertation abstract will be published, please prepare and 
%proofread it carefully. Print all symbols and foreign words clearly and 
%accurately to avoid errors or delays. Make sure that the title given at 
%the top of the abstract has the same wording as the title shown on your 
%title page. Avoid mathematical formulas, diagrams, and other 
%illustrative materials, and only offer the briefest possible description 
%of your thesis/dissertation and a concise summary of its conclusions. Do 
%not include lengthy explanations and opinions.
%The abstract should bear the lower case Roman number ii (if you did not 
%include a copyright page) or iii (if you include a copyright page).

\begin{center}
\vspace*{52pt}
{ABSTRACT}
\vspace{11pt}

\begin{singlespace}
Steven W. Saroka: The Moderating Effects of the Need for Multinational Investment on State Repression \\
(Under the direction of Navin A. Bapat)
\end{singlespace}
\end{center}

What are the effects of resource endowments on state repression? This paper theorizes that states are more likely to engage in repression to secure resource-rich areas to maximize the state’s profits, with repression intensity varying by whether the resource requires outside investment to extract. Should resource extraction require outside investment, states must restrain their repression and share profits with an external multinational corporation. This theory yields predictions that there will be higher levels of state repression closer to sites of natural resources, and that overall levels of repression will be lower in areas with resources that require multinational investment than in areas with resources that the state can extract with its own capabilities. This is tested using a logit model and PRIO-GRID cell-years and repression data from the UCDP's Georeferenced Event Dataset.


\clearpage
